\documentclass[dvipdfmx]{article}

% Language setting
\usepackage[english]{babel}

% Set page size and margins
\usepackage[letterpaper,top=2cm,bottom=2cm,left=3cm,right=3cm,marginparwidth=1.75cm]{geometry}

% Useful packages
\usepackage{amsmath}
\usepackage{amssymb}
\usepackage{amsfonts}
\usepackage{enumerate}
\usepackage{graphicx}
\usepackage{physics}
\usepackage[colorlinks=true, allcolors=blue]{hyperref}

% definition, proposition, lemma, theorem, corollary, remark, fact
\usepackage{amsthm}

\theoremstyle{definition}
\newtheorem{dfn}{Definition}[subsection]
\newtheorem{prop}[dfn]{Proposition}
\newtheorem{lem}[dfn]{Lemma}
\newtheorem{thm}[dfn]{Theorem}
\newtheorem{cor}[dfn]{Corollary}
\newtheorem{rem}[dfn]{Remark}
\newtheorem{fact}[dfn]{Fact}
\renewcommand{\qedsymbol}{$\blacksquare$}


\title{確率解析勉強会第1回}
\author{中津陽}

\begin{document}
\maketitle

\begin{abstract}
 この資料は、「」著「」を元に、確率解析の学習を行うために作成された。
\end{abstract}

\section{一般的な確率論}
\subsection{無限確率空間}
 なぜ$\sigma$-加法族を考える必要があるか。 

\begin{dfn}[$\sigma$-加法族]
    \label{dfn:sigma-alg}
    $\Omega \neq \phi$, $\mathcal{F} \subset 2^{\Omega}$とする。
    $\mathcal{F}$が$\sigma$加法族であるとは、次を満たすことである。
    \begin{enumerate}[(i)]
        \item $\phi \in \mathcal{F}$
        \item $A \in \mathcal{F} \Rightarrow A^{C} \in \mathcal{F}$ 
        \item 
            $
            \forall n \in \mathbb{N} \qty[ A_n \in \mathcal{F} ] \Rightarrow \bigcup_{n=1}^{\infty} A_n \in \mathcal{F}
            $
    \end{enumerate}
\end{dfn}

Definition\ref{dfn:sigma-alg}(i), (iii)より、$A \in \mathcal{F} \land B \in \mathcal{F} \Rightarrow A \cup B \in \mathcal{F}$が成立する。
\begin{proof}
    集合列$A, B, \phi, \phi, \dots$を考える。
    ただし、$A, B \in \mathcal{F}$である。
    Definition\ref{dfn:sigma-alg}(i)より、$\phi \in \mathcal{F}$である。
    したがって、Definition\ref{dfn:sigma-alg}(iii)より、
    $A \cup B = A \cup B \cup \phi \cup \phi \dots \in \mathcal{F}$
    が言える。
\end{proof}

同様にして、ある自然数$N$に対し、$\forall n \in \{1,2,\dots,N \} \qty[A_n \in \mathcal{F}] \Rightarrow \bigcup_{n=1}^{N} A_n \in \mathcal{F}$が成立する。
\begin{proof}
    $N=k$において、題意が満たされるものと仮定する。
    したがって、$\bigcup_{n=1}^{k}A_n \in \mathcal{F}$である。
    $A_{k+1} \in \mathcal{F}$をとる。
    % この時、$\forall n \in \{1,2,\dots,k,k+1 \} \qty[A_n \in \mathcal{F}]$は真である。
    この時、上述の証明より$\bigcup_{n=1}^{k+1}A_n = \bigcup_{n=1}^{k}A_n \cup A_{k+1} \in \mathcal{F}$
    $N=1$の場合は自明なので、以上から再起的に題意は示された。
\end{proof}

Definition\ref{dfn:sigma-alg}(ii),(iii)より、
$\forall n \in \mathbb{N} [A_n \in \mathcal{F}] \Rightarrow \bigcap_{n=1}^{\infty} A_n \in \mathcal{F}$が示される。
\begin{proof}
    
\end{proof}


\end{document}
